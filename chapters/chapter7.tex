\part{Многомерный анализ, интегралы и ряды}

\chapter{Теорема о равномерной непрерывности функции, непрерывной на компакте.}

\section{Теорема о равномерной непрерывности функции, непрерывной на компакте.}

\subsection{Компактные множества.}

\begin{defn}
Множество $G$ точек из $\bbR ^n$ называется \textit{ограниченным}, если существует число $r\ge 0$ такое, что
$$
|OM|\le r \quad \forall M\in G.
$$
Здесь $O$ "--- точка с координатами $(0,0,\dots,0)$.
\end{defn}

\begin{defn}
Точка $M_0$ называется \textit{точкой прикосновения множества $G$}, если в любой её окрестности содержится хотя бы одна точка из $G$.
\end{defn}

\begin{defn}
Множество всех точек прикосновения множества $G$ называется \textit{замыканием} множества $G$ и обозначается $\overline{G}$.
\end{defn}

\begin{defn}
Множество называется \textit{замкнутым}, если оно совпадает со своим замыканием.
\end{defn}

Для любого множества $G$ все его точки и все его предельные точки являются точками прикосновения, и других точек прикосновения нет.

\begin{defn}
Множество $G \subset \bbR^n$ называется \textit{компактным} множеством в $\bbR^n$ (\textit{компактом} в $\bbR^n)$, если $\forall$ последовательности $\{ x_n \}$ точек множества $G$ $\exists$ её подпоследовательность, сходящаяся к точке множества $G$.
\end{defn}

\begin{thm}
Множество $G \subset \bbR^n$ компактно $\Leftrightarrow$
\begin{equation}
\begin{cases}
G \textit{ "--- ограниченное множество} \\
G \textit{ "--- замкнутое множество},
\end{cases}
\end{equation}
\end{thm}

\begin{proof}\leavevmode
\begin{enumerate}[wide, labelwidth=!, labelindent=\parindent]
\item
Докажем, что если $G \subset \bbR^n, \; G$ "--- огранич, $G$ "--- замкнуто, то $G$ "--- компакт.

Пусть $\{ x_k \}$ "--- последовательность точек $\bbR^n: \; x_k \in G, \; \forall k \in \bbN$. Т.к. $G$ "--- ограниченное $\Leftarrow$ то согласно теореме Больцано-Вейерштрасса $\exists \{ x_{k_j} \}$ последовательности $\{ x_k \}$, которая сходится к некоторой точке $x_0 \in \bbR^n$.

Тогда $\forall \epsilon > 0 \; \exists j_0 \in \bbN: \; \forall j > j_0, x_{k_j} \in O_\epsilon(x_0) \cap G \; \Leftarrow \; \forall \epsilon > 0 \; O_\epsilon(x_0) \cap G \not= \emptyset$

$\Rightarrow \; x_0$ "--- точка прикосновения множества $G$

$G$ "--- замкнутое $\Rightarrow \; x_0 \in G$

$\Rightarrow$ Любая последовательность точек $G$ имеет подпоследовательность, сходящуюся к точке $G$ ($G$ "--- компакт)

\item

Докажем, что если множество $G$ не ограничено, то $G$ не является компактом. И докажем, что если множество $G$ не замкнуто, то $G$ не является компактом.

$G$ не является компактом: $\exists$ последовательность $\{ x_n \}$ точек множества $G$ такая, что $\forall$ её подпоследовательность не является сходящейся к точке множества $G$. 
\end{enumerate}

\begin{enumerate}[wide, labelwidth=!, labelindent=\parindent]
\item

$G$ "--- неограниченная, $\forall k \in \bbN \quad \exists x_k \in G\cquad |x_k| > k$

$\exists \{ x_k \}: |x_k| > k \; \Rightarrow \; \forall$ строго монотонно возрастающая натуральная $\{ k_j \}\ \  x_{k_j} > k_j \to +\infty$ при $j \to +\infty \quad \Rightarrow \; x_{k_j}$ расходится, т.е. неограничена 

\item

$G$ "--- не замкнутое, $\exists x_0 \in \bbR^n \;:\; x_0$ "--- предельная точка $G, \; x_0 \notin G \quad \Rightarrow$

$\exists \{ x_k \}: \; \forall k \in \bbN: x_k \in G \setminus \{ x_0 \}, x_k \to x_0$ при $k \to +\infty$

$\Rightarrow \quad \exists \{ x_k \}$ "--- последовательность точек $G$, предел, который $\exists$ и не лежит в $G \quad \Rightarrow$  любая её подпоследовательность не является сходящейся к точке множества $G$.\qedhere
\end{enumerate}
\end{proof}
Область в $\bbR^n$ "--- аналог интервалов в $\bbR^1$

Компакты в $\bbR^n$ "--- аналог отрезков в $\bbR^1$



\subsection{Равномерно непрерывные функции и отображения}

\begin{defn}
Функция $f(M),\;M\in G$, называется \textit{непрерывной на множестве $G$}, если она непрерывна в каждой его точке, т.е. если выполняется условие:
\begin{equation}\label{yaa45e1}
\forall M_0\in G\quad \forall\epsilon>0 \quad \exists\delta>0\cquad\forall M\in G,\quad |MM_0|<\delta\cquad|f(M)-f(M_0)|<\epsilon.
\end{equation}
\end{defn}
Заметим, что здесь $\delta$ зависит как от $\epsilon$, так и от $M_0$.

\begin{defn}
Функция $f(M),\; M\in G$, называется \textit{равномерно непрерывной на множестве G}, если выполняется условие:
\begin{equation}\label{yaa45e2}
\forall\epsilon>0\quad\exists\delta>0\cquad\forall M, M'\in G,\ |MM'|<\delta\cquad |f(M)-f(M')|<\epsilon.
\end{equation}
\end{defn}
Заметим, что здесь $\delta$ зависит только от $\epsilon$.

Очевидно, если выполнено условие \eqref{yaa45e2}, то и выполнено условие \eqref{yaa45e1}, т.е. если функция равномерно непрерывна в любой точке этого множества, то она непрерывна на этом множестве. Как показывают примеры, обратное утверждение является неверным.

\begin{thm}
Если функция $f(M)$ определена и непрерывна на ограниченном замкнутом множестве $G\subset\bbR^n$ (т.е.~$G$ "--- компакт в $\bbR^n$), то она равномерно непрерывна на $G$.
\end{thm}

\begin{proof}
Доказывать будем методом от противного. Предположим, что функция $f(M)$ не является равномерно непрерывной на $G$. Тогда
$$
\exists\epsilon_0>0:\quad\forall\delta>0\quad\exists M,M'\in G:\quad |MM'|<\delta, \quad |f(M)-f(M')|\ge \epsilon_0
$$
Через $M_k$ и $M_k'$ обозначим точки из этого условия, которые соответствуют $\delta=1/k$, т.е. $M_k$ и $M_k'$ принадлежат множеству $G$ и такие, что 
\begin{equation}\label{yaa45e4}
|M_kM_k'|<\frac{1}{k},\quad|f(M_k)-f(M_k')|\ge \epsilon_0.
\end{equation}

Так как множество $G$ ограничено, то последовательность $\{M_k\}$ ограничена, и поэтому по теореме \hyperref[th:ch1:TBV]{Больцано"--~Вейерштрасса} у неё есть сходящаяся подпоследовательность $\{M_{k_p}\}$. Пусть
$$
\lim\limits_{p\to\infty} M_{k_p} =M_0.
$$
Отсюда и из условия $|M_kM_k'|<1/k$ следует, что 
$$
\lim\limits_{p\to\infty} M_{k_p}' =M_0.
$$
А так как множество $G$ замкнуто, то $M_0 \in G$.

Функция $f(M)$ непрерывна в точке $M_0$, поэтому
$$
|f(M_{k_p})-f(M_{k_p}')|\le |f(M_{k_p})-f(M_0)|+|f(M_0)-f(M_{k_p}')|\to 0
$$
при $p\to\infty$, что противоречит условию \eqref{yaa45e4}, которое следует из нашего предположения. Следовательно, это предположение неверное. 

Теорема доказана.
\end{proof}

Эту теорему иногда называют \textit{теоремой Кантора о равномерной непрерывности}\rindex{теорема!Кантора о равномерной непрерывности}. Кратко её формулируют так:

\begin{thmn}
Функция, непрерывная на компакте, равномерно непрерывна.
\end{thmn}

\begin{cons}
Если функция непрерывна на некотором отрезке, то она равномерно непрерывна на этом отрезке. 
\end{cons}















