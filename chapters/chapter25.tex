\chapter{Положительно определённые квадратичные формы. Критерий Сильвестра.}

\section[Положительно определённые квадратичные формы.]{Положительно определённые квадратичные формы\footnote{Рекомендую ознакомиться с написанными самим Чубаровым И.А. материалами по этому билету по этой ссылке: \href{https://drive.google.com/drive/u/0/folders/0BzuzEyNkpwYDcFhhV1l2N1lhY2s}{$https://drive.google.com/drive/...$}}. }
  \begin{defn}
  Квадратичная функция $k(x)$ на линейном пространстве~$L$ (здесь пространство задаётся над полем вещественных чисел: $K=\bbR$) называется \textit{положительно определённой}, если $\forall x \in L$: $x \neq 0 \hookrightarrow k(x)>0;$ \textit{отрицательно определённой}, если $\forall x \in L$: $x \neq 0 \hookrightarrow k(x)<0;$ \\
\textit{неотрицательно определённой (положительно полуопределённой)}, если $\forall x \in L \hookrightarrow k(x)\ge 0;$ \\
\textit{неположительно определённой (отрицательно полуопределённой)}, если $\forall x \in L \hookrightarrow k(x)\le 0;$ \\
\textit{знаконеопределённой}, если $\exists x_1,x_2 \in L$ и $k(x_1)>0$, $k(x_2)<0.$ \\
  \end{defn}
  Пусть в некотором базисе квадратичная функция записана в виде квадратичной формы
  \begin{equation}\label{25.1.common}
  k(x)=b_{11}x_1^2+\ldots+b_{nn}x_n^2+2\sum\limits_{1\le i<j\le n} b_{ij}x_ix_j.
  \end{equation}
с матрицей $B=(b_{ij})$.
  \begin{lemm}\label{25.1.lemm}
  Квадратичная форма тогда и только тогда является положительно определённой, когда она приводится к диагональному виду $\sum\limits_{i=1}^n\alpha_iz_i^2, \alpha_i>0, i=\overline{1,n}$ или, что равносильно, каноническому виду $\sum\limits_{i=1}^ny_i^2$.
  \end{lemm}
  \begin{notion}
  От диагонального вида можно прйти к каноническому при помощи замены $y_i=\sqrt{\alpha_i}z_i$.
  \end{notion}
  \begin{proof}
  То, что диагональная форма со всеми положительными коэффициентами положительно определена, очевидно. \\
  Обратно, допустим, что данная положительно определённая форма приводится к виду $\sum\limits_{i=1}^py_i^2-\sum\limits_{i=p+1}^{p+q}y_i^2$. Если, вопреки доказываемому, $p<n$, то $k(0,0,...,1)\le 0$. Получаем противоречие с уловием.
  \end{proof}
\section{Критерий Сильвестра}
  \begin{thm} [Критерий Сильвестра\rindex{критерий!Сильвестра}]
  Для положительной определённости квадратичной формы $k(x)$ в $\bbR^n$ необходимо и достаточно, чтобы все \underline{главные миноры} её матрицы $B$, имеющие вид
  \begin{equation}
  \Delta_m=\det \begin{Vmatrix}
  b_{11} & b_{12} & \cdots & b_{1m} \\
  b_{21} & b_{22} & \cdots & b_{2m} \\
  \vdots & \vdots & \ddots & \vdots \\
  b_{m1} & b_{m2} & \cdots & b_{mm} \\
  \end{Vmatrix}(b_{ij}=b_{ji} \forall i,j), m=\overline{1,n},
  \end{equation}
были положительными
  \end{thm}
  \begin{proof} \leavevmode
  \begin{itemize}[wide, labelwidth=!, labelindent=\parindent]
  \item[\fbox{дост.:}] Пусть дано, что все главные миноры матрицы квадратичной формы положительны, и надо доказать, что она является положительно определённой. Воспользуемся методом математической индукции и леммой~\ref{25.1.lemm}. 

  Случай для \underline{$n=1$} очевиден. 
  
  Допустим, что \underline{$n>1$} и из положительности главных миноров матрицы квадратичной формы вплоть до $(n-1)$-го порядка включительно следует возможность приведения квадратичной формы от $n-1$ переменных $x_1,...,x_{n-1}$ к виду $k(x)=\sum\limits_{i=1}^{n-1}x_i^2$. Покажем, что достаточность имеет место и в случае $n$ переменных.
  
  В выражении квадратичной формы от переменных $x_1,...,x_n$ выделим слагаемые, содержащие $x_n$:
  \begin{equation*}
  k(x)=\sum_{j=1}^{n-1}\sum_{i=1}^{n-1}b_{ij}x_jx_i+2\sum_{j=1}^{n-1}b_{jn}x_jx_n+b_{nn}x_n^2.
  \end{equation*}
  Двойная сумма $\sum\limits_{j=1}^{n-1}\sum\limits_{i=1}^{n-1}b_{ij}x_jx_i=k^*(x)$ есть квадратичная форма, зависящая от $n-1$ переменной, причём главные миноры её матрицы совпадают с главными минорами $k(x)$ до порядка $n-1$ включительно, которые, по условию, положительны. Отсюда следует, по предположению индукции, что квадратичная форма $k^*(x)$ положительно определена и для неё существует невырожденная замена переменных $x_j=\sum\limits_{i=1}^{n-1}\sigma_{ji}y_i, j=\overline{1,n-1}$, приводящая её к каноническому виду $k^*(x)=\widetilde k^*(y)=\sum\limits_{i=1}^{n-1}y_i^2$.
  
  Запишем квадратичную форму в новых переменных:
  \begin{equation*}
  k(x)=\widetilde k(y_1,...,y_{n-1},x_n)=\sum\limits_{i=1}^{n-1}y_i^2+2\sum_{j=1}^{n-1}b'_{jn}y_jx_n+b_{nn}x_n^2
  \end{equation*}
и выделим полные квадраты по $y_1,...,y_{n-1}$:
  \begin{equation*}\begin{array}{crl}
  k=\sum\limits_{i=1}^{n-1}(y_i^2+2b'_{in}y_ix_n+{b'_{in}}^2x_n^2)+(b_{nn}-\sum\limits_{i=1}^{n-1}{b'_{in}}^2)x_n^2 =\\
  =\sum\limits_{i=1}^{n-1}z^2+b''_{nn}x_n^2,
  \end{array}\end{equation*}
где введено обозначение $ b''_{nn}=b_{nn}-\sum\limits_{i=1}^{n-1}{b'_{in}}^2 $ и произведена замена $z_i=y_i+b'_{in}x_n, i=\overline{1,n-1}, x_n=x_n$. Эта замена, очевидно, невырожденная.

  Теперь вспомним, что определитель матрицы квадратичной формы сохраняет знак при замене базиса. По условию определитель матрицы $B$ квадратичной функции в исходном базисе положительный, поскольку является главным минором порядка $n$. Но из выражения для $k(x)$ в конечном базисе мы получаем, что определитель матрицы квадратичной формы $k$ равен $b''_{nn}$. Поэтому $b''_{nn}>0$ и можно ввести переменную $z_n=\sqrt{b''_{nn}}x_n$, в результате чего получаем канонический вид: $k=\sum\limits_{i=1}^nz_i^2$.
  
  \item[\fbox{необх.:}] Дано, что квадратичная функция положительно определена, и надо доказать положительность главных миноров её матрицы. Снова применим индукцию по числу переменных $n$. 
  
  Для \underline{n=1} это ясно.
  
  Пусть \underline{n>1} и для форм от меньшего числа переменных утверждение теоремы верно. Поскольку квадратичная форма $k^*(x)=\sum\limits_{j=1}^{n-1}\sum\limits_{i=1}^{n-1}b_{ij}x_jx_i$ является положительно определённой (её значения –- это значения $k(x)$ при $x_n=0$), то по предположению индукции её главные миноры, совпадающие с главными минорами матрицы $B$ до порядка ${n-1}$, положительны. А определитель самой матрицы $B$, который является главным минором порядка $n$, положителен, поскольку $k(x)$ приводится к каноническому виду $k=\sum\limits_{i=1}^nz_i^2$, и определитель матрицы полученной при этом квадратичной формы равен 1 и имеет такой же знак, как и определитель матрицы $B$. 
  \end{itemize}
  \end{proof}
  
  \begin{cons} [Критерий Сильвестра для отрицательной определённости]
  Для отрицательной определённости квадратичной формы  $k(x)$ в $\bbR^n$ необходимо и достаточно, чтобы все главные миноры её матрицы $B$ имели чередующиеся знаки, начиная с минуса, т.е. $(-1)^m\Delta_m>0,m=\overline{1,n}$.
  \end{cons}
  \begin{proof}
  Рассмотрим форму $-k(x)$ c матрицей $B'=-B=(-b_{ij})$: её положительной определённости, по критерию Сильвестра, равносильно условие
  \begin{equation}
  \Delta_m=\det \begin{Vmatrix}
  -b_{11} & -b_{12} & \cdots & -b_{1m} \\
  -b_{21} & -b_{22} & \cdots & -b_{2m} \\
  \vdots  & \vdots  & \ddots & \vdots \\
  -b_{m1} & -b_{m2} & \cdots & -b_{mm} \\
  \end{Vmatrix}=(-1)^m\Delta_m>0, m=\overline{1,n}.
  \end{equation}
  \end{proof}