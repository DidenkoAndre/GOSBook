\chapter[Свойства интеграла с переменным верхним пределом (непрерывность, дифференцируемость). Формула Ньютона"--~Лейбница.]{Свойства интеграла с переменным верхним пределом (непрерывность, дифференцируемость). Формула Ньютона"--~Лейбница.}
\section{Определение интеграла Римана}

\begin{defn}
\textit{Разбиением промежутка} $\Delta$ называется любое конечное множество $\tau(\Delta) = \{ \Delta_1, \ldots, \Delta_N\}$ попарно непересекающихся промежутков $\Delta_1, \ldots, \Delta_N$, объединение которых равно $\Delta$.

Если $a$ и $b$ "--- концы промежутка  $\Delta$, а $x_{i-1}$ и $x_{i}$ — концы промежутка $\Delta_i$, то
$$
a=x_0\le x_1\ldots \le x_n=b.
$$
Множество точек $\{x_0,x_1,\dots,x_n\}$ называется \textit{точками разбиения} $\tau(\Delta)$ промежутка $\Delta$.
\end{defn}

Пусть на конечном промежутке $\Delta$ задана функция $f(x)$, и пусть $\tau(\Delta) = \{ \Delta_1, \ldots, \Delta_N\}$ "--- некоторое разбиение промежутка $\Delta$. 
Через $m_i$ и $M_i$ обозначим точные грани функции $f$ на промежутке $\Delta_i$:
$$
m_i = \inf_{x \in \Delta_i} f(x), \quad M_i = \sup_{x \in \Delta_i} f(x),
$$ 
 а через $|\Delta_i|$ "--- длину промежутка $\Delta_i$. Очевидно, что если $x_{i-1}$ и $x_i$ "--- концы промежутка $\Delta_i$, то $|\Delta_i|=x_i-x_{i-1}$, поэтому вместо $|\Delta_i|$ иногда будем писать $\Delta x_i$. 

\begin{defn}
Для функции $f(x), \ x \in \Delta$, и разбиения $\tau(\Delta) = \{ \Delta_1, \ldots, \Delta_N\}$ суммы 
$$
\sum_{i = 1}^{N}m_i|\Delta_i| \quad \text{и} \quad \sum_{i = 1}^{N}M_i|\Delta_i|
$$
называются \textit{интегральными суммами Дарбу} (соответственно, \textit{нижней и верхней}) и обозначаются $s(f; \tau)$ и $S(f; \tau)$.
\end{defn}

Очевидно, что для $\forall f(x), \ x \in \Delta$, и $\forall \tau(\Delta)$ справедливо неравенство 
$$
s(f; \tau) \le S(f; \tau).
$$

\begin{defn}
Пусть задана функция $f(x), \ x \in \Delta$, и некоторое разбиение $\tau(\Delta) = \{ \Delta_1, \ldots, \Delta_N\}$ промежутка $\Delta$. Тогда сумма
$$
\sum_{i = 1}^{N} f(\xi_i)|\Delta_i|,
$$
где $\xi_i \in \Delta_i$ называется \textit{интегральной суммой Римана функции $f$} и обозначается $\sigma(f; \tau)$ или $\sigma(f; \tau; \xi)$, где $\xi = (\xi_1, \ldots, \xi_N)$. $\xi$ будем называть выборкой точек, подчинённой разбиению $\tau$.
\end{defn}

Очевидно, для любого разбиения $\tau(\Delta) = \{ \Delta_1, \ldots, \Delta_N\}$ справедливы неравенства
$$
s(f; \tau) \le \sigma(f; \tau; \xi) \le S(f; \tau)
$$
при любом выборе точек $\xi_i \in \Delta_i$. Кроме того,
$$
s(f; \tau) = \inf_{\xi} \sigma(f; \tau; \xi), \quad S(f; \tau) = \sup_{\xi} \sigma(f; \tau; \xi).
$$

\begin{defn}
Точные грани $\sup\limits_{\tau} s(f; \tau)$ и $\inf\limits_{\tau} S(f; \tau)$ называются \textit{интегралами Дарбу} (соответственно, \textit{нижним и верхним}) \textit{от функции $f$ по промежутку $\Delta$} и обозначаются $\underline{J}(f)$ и $\overline{J}(f)$. Таким образом,
$$
\underline{J}(f) = \sup_{\tau} s(f; \tau), \quad \overline{J}(f) = \inf_{\tau} S(f; \tau),
$$
причём $\underline{J}(f) \le \overline{J}(f)$ для любой функции $f$, определённой на конечном промежутке 
$\Delta$.
\end{defn}


\begin{defn}
Если интегралы Дарбу от функции $f$ по промежутку $\Delta$ конечны и равны между собой, то функция $f$ называется \textit{интегрируемой по Риману на промежутке $\Delta$}, а число
$$
J(f) = \underline{J}(f) = \overline{J}(f)
$$
называется \textit{интегралом Римана от функции $f$ по промежутку $\Delta$} и обозначается $\int_{\Delta} f(x) \,dx$.
\end{defn}

Из этого определения следует, что если функция $f$ интегрируема на промежутке $\Delta$, то
$$
s(f; \tau) \le \int_{\Delta} f(x) \,dx \le S(f; \tau) \quad \forall \tau(\Delta).
$$

Если $\Delta = [a, b]$, интеграл обозначают $$\int_{a}^{b} f(x) \,dx.$$

\begin{defn}
Пусть $\tau(\Delta) = \{ \Delta_1, \ldots, \Delta_N\}$ "--- некоторое разбиение конечного промежутка $\Delta$. Тогда число, равное $\max\limits_i|\Delta_i|$, где $i=1,\ldots,N$, называется \textit{мелкостью разбиения $\tau$} и обозначается $|\tau|$.
\end{defn}

\begin{defn}
Последовательность $\{\tau_k\}$ разбиений промежутка $\Delta$ называется \textit{римановой}, если $\lim\limits_{k \to +\infty}|\tau_k|=0$.
\end{defn}

\begin{thm} \label{ch11.1thm1}
Если функция f интегрируема на промежутке $\Delta \subset D_f$, то для любой римановой последовательности разбиений $\tau_k(\Delta),\ k\in \bbN$, имеют место равенства
\begin{equation} \label{ch11.1eq1}
\lim_{|\tau| \to 0} \sigma(f, \tau, \xi) = \lim_{k \to \infty} \sigma(f, \tau_k, \xi) = \int_{\Delta} f(x) \,dx.
\end{equation}
\end{thm}

\section{Определённый интеграл как функция верхнего (нижнего) предела.}

Для любой функции $f$, определённой и интегрируемой на промежутке~$\Delta$, функция
\begin{equation} \label{ch11.2eq1}
F(x) = \int_{c}^{x} f(t) \,dt, \quad x \in \Delta,
\end{equation}
где $c \in \Delta$, называется \textit{интегралом с переменным верхним пределом}, а функция
\begin{equation} \label{ch11.2eq2}
\Phi(x) = \int_{x}^{c} f(t) \,dt, \quad x \in \Delta,
\end{equation}
"--- \textit{интегралом с переменным нижним пределом}.

Очевидно, $\Phi(x) = -F(x)\ \forall x \in \Delta$. Поэтому ограничимся рассмотрением только интегралов с переменным верхним пределом.

\begin{thm}
Если функция $f$ интегрируема на промежутке $\Delta$, то функция $\eqref{ch11.2eq1}$ на $\Delta$ удовлетворяет условию
$$
|F(x_2) - F(x_1)| \le ||f|| \cdot |x_2 - x_1| \quad \forall x_1, x_2 \in \Delta,
$$
где $||f|| = \sup\limits_{x \in \Delta} |f(x)|$.
\end{thm}

\begin{proof}
Действительно, $\forall x_1, x_2$ из $\Delta$ имеем:
\begin{multline*}
|F(x_2) - F(x_1)| = \left| \int_{c}^{x_2}f(t) \,dt - \int_{c}^{x_1} f(t) \,dt\right| = \left| \int_{x_1}^{x_2}f(t) \,dt \right| \le\\ \le \left| \int_{x_1}^{x_2}|f(t)| \,dt \right| \le ||f|| \cdot |x_2 - x_1|.\tag*{\qedhere}
\end{multline*}
\end{proof}

\begin{cons}
Если функция $f$ интегрируема на промежутке $\Delta$, то функции $\eqref{ch11.2eq1}$ и $\eqref{ch11.2eq2}$ непрерывны на $\Delta$.
\end{cons}

\begin{thm}
Если функция $f$ интегрируема на промежутке $\Delta$ и непрерывна в точке $x_0 \in \Delta$, то функция $\eqref{ch11.2eq1}$ в точке $x_0$ имеет производную и $F'(x_0) = f(x_0)$.
\end{thm}

\begin{proof}
Пусть $x \in \Delta$, $\Delta x = x - x_0$ и $\Delta F = F(x) - F(x_0)$. Тогда, если $x \not= x_0$, то 
$$ 
\frac{\Delta F}{\Delta x} = \frac{1}{\Delta x} \int_{x_0}^{x} f(t) \,dt, \quad f(x_0) = \frac{1}{\Delta x} \int_{x_0}^{x} f(x_0) \,dt,
$$
и поэтому 

\begin{equation} \label{ch11.2eq3}
\left| \frac{\Delta F}{\Delta x} - f(x_0) \right| = \frac{1}{|\Delta x|} \left| \int_{x_0}^{x} \left( f(t) - f(x_0) \right) \,dt \right|.
\end{equation}

Так как функция $f$ непрерывна в точке $x_0$, то
$$
\forall \epsilon > 0 \quad \exists \delta > 0 : \quad \forall x \in O_\delta(x_0) \cap \Delta \quad |f(x) - f(x_0)| < \epsilon.
$$

Отсюда из равенства $\eqref{ch11.2eq3}$ следует, что 
$$
\left| \frac{\Delta F}{\Delta x} - f(x_0) \right| \le \frac{1}{|\Delta x|} \left| \int_{x_0}^{x} \epsilon \,dt\right| = \epsilon
$$
$\forall x \in \overset{\circ}{O}_\delta(x_0) \cap \Delta$, и поэтому $F'(x_0)$  существует и $F'(x_0) = f(x_0).$
\end{proof}

\begin{defn}
Функция $F(x)$ называется \textit{первообразной} (или \textit{точной первообразной}) для функции $f(x)$ на промежутке  $\Delta$, если $F(x)$ дифференцируема на $\Delta$ и
$$
F'(x) = f(x) \quad \forall x \in \Delta.
$$

Очевидно, если $F(x)$ "--- первообразная для $f(x)$, то и любая функция вида $F(x) + C$,  где $C$ "--- произвольная постоянная, будет первообразной для $f(x)$. 
\end{defn}

\begin{cons}
Пусть функция $f(x)$ интегрируема по Риману и непрерывна на промежутке $\Delta$. Тогда $\forall c \in \Delta$ функция $\eqref{ch11.2eq1}$ непрерывно дифференцируема и $\forall x \in \Delta$ $F'(x_0) = f(x_0)$. 
\end{cons}
У  любой непрерывной, интегрируемой по Риману на промежутке $\Delta$ функции $f$ есть первообразная $\eqref{ch11.2eq1}$.

\section[Формула Ньютона"--~Лейбница]{Формула Ньютона"--~Лейбница\rindex{формула!Ньютона---Лейбница}}

\begin{thm}
Если функция $f(x)$ на отрезке $[a; b]$ интегрируема и имеет первообразную $F(x)$, то
\begin{equation} \label{ch11.3eq1}
\int_{a}^{b} f(x) \,dx = F(b) - F(a).
\end{equation}
\end{thm}

\begin{proof}
Пусть сначала функция $F(x)$ непрерывна на отрезке $[a;b]$, дифференцируема на интервале $(a;b)$ и $F'(x) = f(x) \quad \forall x \in (a;b)$. Тогда для любых точек $x_0$,~$x_1$, \ldots,~$x_N$, таких, что
$$
a = x_0 < x_1 < \ldots < x_n = b,
$$
имеем
$$
F(b) - F(a) = \sum_{i = 1}^{N} \left( F(x_i) - F(x_{i - 1}) \right).
$$

На любом отрезке $[x_{i - 1}, x_i]$ функция $F(x)$ удовлетворяет всем условиям теоремы Лагранжа о среднем, поэтому
$$
\exists \xi_i \in (x_{i - 1}; x_i) \cquad F(x_i) - F(x_{i - 1}) = f(\xi_i) \Delta x_i,
$$
и, следовательно,
$$
F(b) - F(a) = \sum_{i = 1}^{N} f(\xi_i) \Delta x_i = \sigma(f;\tau;\xi),
$$
где $\tau$ "--- разбиение отрезка $[a;b]$ точками $x_0$,~$x_1$, \ldots,~$x_N$. А так как функция $f(x)$ интегрируема на $[a;b]$, то по теореме~\ref{ch11.1thm1}
$$
\lim_{|\tau| \to 0} \sigma (f;\tau; \xi) = \int_{a}^{b} f(x) \,dx.
$$

Формула $\eqref{ch11.3eq1}$ доказана в случае, когда $F'(x) = f(x)$ на интервале $(a;b)$. В общем случае, пусть функция $F(x)$ непрерывна на отрезке $[a;b]$ и $F'(x) = f(x)$ всюду, кроме, может быть, конечного числа точек $c_0, c_1, \ldots, c_N$, таких, что $a = c_0 < c_1 < \ldots < c_N = b$, где $F'(x) \not= f(x)$ или $F'(x)$ не существует. Тогда, как уже доказано,
$$
\int_{c_{i - 1}}^{c_i} f(x) \,dx = F(c_i) - F(c_{i - 1}),
$$
и поэтому
\begin{equation*}
\int_{a}^{b} f(x) \,dx = \sum_{i = 1}^{N} \int_{c_{i - 1}}^{c_{i}} f(x) \,dx = F(b) - F(a). \tag*{$\qedhere$}
\end{equation*}
\end{proof}


Формула~\eqref{ch11.2eq1} называется \textit{формулой Ньютона"--~Лейбница}. В ней вместо разности $F(b) - F(a)$ иногда пишут $F(x)\big|_{a}^{b}$, и тогда она принимает вид
$$
\int_{a}^{b} f(x) \,dx = F(x)\bigg|_{a}^{b}.
$$

\begin{cons}
Если функция $f(x)$ на отрезке $[a;b]$ интегрируема и имеет первообразную $F(x)$ (точную или обобщённую), то
$$
F(x) = F(a) + \int_{a}^{x} f(t)\,dt.
$$
\end{cons}

\begin{cons}
Если функция $f(x)$ на отрезке $[a;b]$ интегрируема и имеет точную первообразную, то существует $\xi \in (a;b)$ такое, что

\begin{equation} \label{ch11.3eq2}
\int_{a}^{b}f(x) \,dx = f(\xi) (b - a).
\end{equation}

\end{cons}
\begin{proof}
Действительно, формула~\eqref{ch11.3eq2} следует из формулы~\eqref{ch11.3eq1} и формулы Лагранжа о среднем для функции $F(x)$.
\end{proof}

В заключение доказанную теорему сформулируем как теорему об интеграле от производной:

\textit{Если функция $F(x)\colon x \in [a;b]$, на отрезке $[a;b]$ непрерывна и кусочно-дифференцируема, а её производная интегрируема на $[a;b]$, то}
\begin{equation} \label{ch11.3eq3}
\int_{a}^{b}F'(x) \,dx = F(b) - F(a).
\end{equation} 

Формула~\eqref{ch11.3eq3}, как и формула~\eqref{ch11.2eq1}, называется формулой Ньютона"--~Лейбница.